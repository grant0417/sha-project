\documentclass{IEEEtran}

\usepackage[T1]{fontenc} \usepackage[utf8]{inputenc}

\pdfminorversion=7

% Additional Math Support
\usepackage{amsmath, amssymb, amsthm}
% Inserting Graphics
\usepackage{graphicx}
% Better Tables
\usepackage{booktabs}
% Inserting URLs
\usepackage{url}
% Makes paragraphs distinct from skips
\usepackage{parskip}
% Hide page number when page is empty
\usepackage{emptypage}

\usepackage[bookmarks]{hyperref}

\usepackage{xcolor}
\usepackage{listings}
% RISC-V Assembler syntax and style for latex lstlisting package
% 
% These are risc-v commands as per our university (University Augsburg, Germany) guidelines.
%
% Author: Anton Lydike
%
% This code is in the public domain

% language definition

\definecolor{codegreen}{rgb}{0,0.6,0}
\definecolor{codegray}{rgb}{0.5,0.5,0.5}
\definecolor{codepurple}{rgb}{0.58,0,0.82}

\lstdefinelanguage[RISC-V]{Assembler}
{
  alsoletter={.}, % allow dots in keywords
  alsodigit={0x}, % hex numbers are numbers too!
  morekeywords=[1]{ % instructions
    lb, lh, lw, lbu, lhu,
    sb, sh, sw,
    sll, slli, srl, srli, sra, srai,
    add, addi, sub, lui, auipc,
    xor, xori, or, ori, and, andi,
    slt, slti, sltu, sltiu,
    beq, bne, blt, bge, bltu, bgeu,
    j, jr, jal, jalr, ret,
    scall, break, nop,
    mul, mulh, mulhsu, div, rem, divu,
    flw
  },
  morekeywords=[2]{ % sections of our code and other directives
    .align, .ascii, .asciiz, .byte, .data, .double, .extern,
    .float, .globl, .half, .kdata, .ktext, .set, .space, .text, .word
  },
  morekeywords=[3]{ % registers
    x0, x1, x2, x3, x4, x5, x6, x7, x8, x9, x10, x11, x12, x13, x14,
    x15, x16, x17, x18, x19, x20, x21, x22, x23, x24, x25, x26, x27,
    x28, x29, x30, x31,
    zero, ra, sp, gp, tp, s0, fp,
    t0, t1, t2, t3, t4, t5, t6,
    s1, s2, s3, s4, s5, s6, s7, s8, s9, s10, s11,
    a0, a1, a2, a3, a4, a5, a6, a7,
    ft0, ft1, ft2, ft3, ft4, ft5, ft6, ft7,
    fs0, fs1, fs2, fs3, fs4, fs5, fs6, fs7, fs8, fs9, fs10, fs11,
    fa0, fa1, fa2, fa3, fa4, fa5, fa6, fa7,
    f0, f1
  },
  morecomment=[l]{;},   % mark ; as line comment start
  morecomment=[l]{\#},  % as well as # (even though it is unconventional)
  morestring=[b]",      % mark " as string start/end
  morestring=[b]'       % also mark ' as string start/end
}

% usage example:

% define some basic colors
\definecolor{mauve}{rgb}{0.58,0,0.82}

\lstset{
  basicstyle=\footnotesize\ttfamily,                    % very small code
  breaklines=true,                              % break long lines
  commentstyle=\color{codegray},  % comments are green
  keywordstyle=[1]\color{blue!80!black},        % instructions are blue
  numberstyle=\tiny\color{codegray},
  keywordstyle=[2]\color{orange!80!black},      % sections/other directives are orange
  keywordstyle=[3]\color{red!50!black},         % registers are red
  stringstyle=\color{mauve},                    % strings are from the telekom
  identifierstyle=\color{teal},                 % user declared addresses are teal
  language=[RISC-V]Assembler,                   % all code is RISC-V
  tabsize=4,                                    % indent tabs with 4 spaces
  showstringspaces=false,                        % do not replace spaces with weird underlines
  numbers=left,
  numbersep=5pt,
}


\title{RISC V Implementation of SHA-3} \author{Grant Gurvis}

\begin{document}

\maketitle

\begin{abstract} This is the abstract \end{abstract}

\section{Introduction}

Cryptographic hash functions are a fundamental primitive in modern cryptography,
they allow mapping of messages to a digest with no know way to reverse other
than brute-force search. Among the most famous hash functions are the Secure
Hash Algorithms (SHA) family of hash functions, the most recent standard being
the SHA-3 algorithm. SHA-3 is amoung the most secure hash functions 

\section{Background of Algorithm} The SHA family of algorithms has been the
standard hash functions used in most cryptographic applications since they were
standardized. SHA-1 and SHA-2 are both based on the Merkle–Damgård construction,
which is also used by other hash functions such as MD5. SHA-1 has been shown to
no longer be secure several time, in 2015 practical attacks were shown to be
possible,\cite{fischlin_freestart_2016} in 2017 an actual collision was
found,\cite{stevens_first_2017} and in 2019 chosen-prefix attacks were shown to
be practical.\cite{ishai_collisions_2019} Since the attacks on SHA-1, many
systems have migrated to SHA-2, such as TLS and Git. SHA-2 is currently used in
wide use but some holes in its security have been shown since it uses similar
underlying mathematics as SHA-1. For example all Merkle–Damgård construction are
susceptible to length extension attacks and collision attacks on SHA-2 are an
active field of research where attacks aren't practical, but have been getting
closer. This is not of immediate concern but a transition to SHA-3 would help
mitigate potential weaknesses of SHA-2 that are found in the future.

SHA-3 is based on the Keccak algorithm which was designed by Guido Bertoni, Joan
Daemen,  Michael Peeters, and Gilles Van Assche. It is the first SHA to not be
designed by the National Security Agency (NSA) which has led to some increased
confidence in its security since the NSA has potentially previously created
compromised algorithms.\cite{perlroth_government_2013} The algorithm was chosen
to be the third SHA after it was chosen in a 2012 public competition run by the
National Institute of Standards and Technologies (NIST) who maintain many
cryptographic and technological standards (including SHA). SHA-3 is known to be
slower in software implementations than previous SHAs but it is much faster when
implemented via hardware accelerators. 

The current standard for SHA-3 was published in 2015.\cite{dworkin_sha-3_2015}

\section{Code Details}

This implement

\section{Conclusion}

\bibliographystyle{IEEEtran} \bibliography{project}

\section*{Appendix}

Hello there


\begin{center}
\begin{lstlisting}
myfunc:
    addi    sp, sp, -8
    sd      ra, 0(sp)
    la      a0, prompt
    mv      a1, t0
    mv      a2, t1
    call    printf
    ld      ra, 0(sp)
    addi    sp, sp, 8
    ret
\end{lstlisting}
\end{center}

\end{document}

